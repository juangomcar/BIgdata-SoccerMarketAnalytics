\documentclass[12pt,letterpaper]{article}
\usepackage[utf8]{inputenc}
\usepackage[spanish]{babel}
\usepackage{geometry}
\usepackage{graphicx}
\usepackage{hyperref}
\usepackage{setspace}
\usepackage{fancyhdr}
\usepackage{booktabs}
\usepackage{enumitem}

\geometry{margin=1in}
\setstretch{1.5}

% =======================
% Portada
% =======================
\begin{document}
\begin{titlepage}
    \centering
    {\Large \textbf{Universidad de La Sabana}}\\[0.3cm]
    {\large Facultad de Ingeniería}\\[0.3cm]
    {\large Maestría en Analítica Aplicada (Coterminal)}\\[2cm]

    {\LARGE \textbf{Proyecto Final – Herramientas de Big Data}}\\[0.5cm]
    {\Large Predicción del Valor de Mercado de Futbolistas}\\[2cm]

    \textbf{Autores:}\\
   Esteban Bernal - C´od. 271930
    Juan Montes - C´od. 272113
    Juan Gomez - C´od. 286774z
 \\[0.5cm]

    \textbf{Asignatura:} Herramientas de Big Data (Coterminal)\\
    \textbf{Profesor:} Hugo Franco, Ph.D. \\[2cm]

    Chía, Colombia \\
    Septiembre 5, 2025

    \vfill
\end{titlepage}

% Índice

\tableofcontents
\newpage

% Resumen Ejecutivo

\section{Resumen Ejecutivo (Brief)}
Este proyecto presenta el diseño e implementación de un pipeline de datos bajo el paradigma ETL (Extract, Transform, Load) para la predicción del valor de mercado de futbolistas profesionales. 
Se integraron fuentes de datos como Transfermarkt (valores históricos) y FBref (métricas de rendimiento), procesados mediante Python y orquestados en un entorno reproducible con PostgreSQL y Docker. 
El objetivo fue entrenar modelos de Machine Learning (Regresión Lineal, Random Forest y XGBoost), demostrando que es posible obtener estimaciones objetivas y transparentes frente a los valores de referencia del mercado.

% Abstract


\section{Abstract}
This project presents the design and implementation of a data pipeline under the ETL (Extract, Transform, Load) paradigm for predicting the market value of professional football players. 
We integrated multiple data sources, such as Transfermarkt (historical values) and FBref (performance metrics), processed with Python and orchestrated in a reproducible environment using PostgreSQL and Docker. 
The goal was to train Machine Learning models (Linear Regression, Random Forest, and XGBoost), showing that it is possible to generate objective and transparent estimates compared to market reference values.


% Introducción
% =======================
\section{Introducción}
\subsection{Formulación del problema}
La valoración de jugadores de fútbol profesional es clave para clubes, agentes y patrocinadores. Sin embargo, los valores reportados en plataformas como Transfermarkt carecen de transparencia metodológica. Esto genera incertidumbre y riesgos en decisiones de inversión deportiva.

\subsection{Marco conceptual}
\begin{itemize}
    \item \textbf{ETL Pipeline}: extracción, transformación y carga de datos.
    \item \textbf{Métricas Avanzadas}: indicadores como goles esperados (xG), asistencias esperadas (xA).
    \item \textbf{EDA}: análisis exploratorio de datos para identificar patrones.
    \item \textbf{Modelos}: comparación de regresión lineal, Random Forest y XGBoost.
\end{itemize}

\subsection{Antecedentes}
Diversos estudios muestran correlación entre rendimiento ofensivo y valor de mercado, aunque suelen limitarse a ligas específicas o datasets pequeños. Nuestro enfoque amplía la escala y aplica mejores prácticas de reproducibilidad.

\subsection{Objetivo}
Desarrollar un pipeline reproducible que integre datos de mercado y rendimiento para entrenar modelos de predicción robustos del valor de futbolistas.

% =======================
% Datos empleados
% =======================
\section{Datos empleados}
\subsection{Fuentes de datos}
\begin{enumerate}
    \item \textbf{FBref.com}: vía web scraping para estadísticas de rendimiento (goles, asistencias, xG, xA, tackles).
    \item \textbf{Transfermarkt (Kaggle)}: dataset histórico de valores de mercado, datos contractuales y demográficos.
\end{enumerate}

\subsection{Contenido y variables}
\begin{itemize}
    \item \textbf{Dependiente:} Valor de mercado en euros.
    \item \textbf{Independientes:} Edad, posición, nacionalidad, goles/90, asistencias/90, xG, xA, tackles, club, liga.
\end{itemize}

% =======================
% Metodología
% =======================
\section{Materiales y Métodos}
\subsection{Pipeline ETL}
\begin{enumerate}
    \item \textbf{Extracción:} Scraping de FBref y carga de Transfermarkt desde Kaggle.
    \item \textbf{Transformación:} Limpieza, unificación y normalización de datos.
    \item \textbf{Carga:} Generación de un dataset consolidado (\texttt{dataset\_ready.csv}) almacenado en PostgreSQL vía Docker.
\end{enumerate}

\subsection{Análisis Exploratorio (EDA)}
Distribuciones, correlaciones de Pearson, boxplots por posición y outliers identificados.

\subsection{Modelado}
Comparación de regresión lineal, Random Forest y XGBoost con métricas MAE, RMSE y $R^2$.

% =======================
% Resultados
% =======================
\section{Resultados}
\begin{itemize}
    \item XGBoost obtuvo el mejor desempeño predictivo.
    \item Se identificaron jugadores sobrevalorados e infravalorados respecto a Transfermarkt.
    \item Predicciones para 2025/2026 muestran consolidación de estrellas jóvenes como Bellingham y Yamal.
\end{itemize}

% =======================
% Discusión y Conclusiones
% =======================
\section{Discusión y Conclusiones}
\subsection{Discusión}
Los resultados validan que el mercado tiende a sobrevalorar talento ofensivo y jóvenes promesas, mientras que los jugadores veteranos muestran ajustes a la baja. 
El modelo se centra en métricas objetivas, excluyendo factores externos (marketing, popularidad).

\subsection{Conclusiones}
\begin{itemize}
    \item Se construyó un pipeline reproducible en Python con PostgreSQL y Docker.
    \item El modelo XGBoost ofrece estimaciones más objetivas que Transfermarkt.
    \item El sistema puede ampliarse con datos de lesiones o redes sociales para futuras versiones.

\end{itemize}

% =======================
% Bibliografía
% =======================
\section{Bibliografía}
\begin{enumerate}
    \item Franco, H. (2025). \textit{Material de clase – Herramientas de Big Data}. Universidad de La Sabana.
    \item McKinney, W. (2017). \textit{Python for Data Analysis}. O'Reilly Media.
    \item James, G., Witten, D., Hastie, T., \& Tibshirani, R. (2013). \textit{An Introduction to Statistical Learning}. Springer.
\end{enumerate}

% =======================
% Anexos
% =======================
\section{Anexos}
\subsection{Código Fuente}
El código fuente completo (pipelines ETL, notebooks de modelado y consultas) está disponible en el repositorio de GitHub:  
\url{https://github.com/usuario/bigdata-futbol-valor-mercado}

\subsection{Datos empleados}
Los datos crudos y el dataset procesado final (\texttt{dataset\_ready.csv}) se encuentran en el directorio \texttt{/data} del repositorio.

\end{document}




